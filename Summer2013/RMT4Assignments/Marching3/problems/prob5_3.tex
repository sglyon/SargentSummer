%!TEX root = ../marching3.tex

\begin{homeworkProblem}[Problem 5.3]

  A household chooses $\{c_t, a _{t+1}\}_{t=0}^{\infty}$ to maximize

  \begin{align} \label{eq:p3e1}
    - \sum_{t=0}^{\infty} \beta^t \left\{ (c_t - b)^2 + \gamma i_t^2 \right \}
  \end{align}

  subject to

  \begin{align} \label{eq:p3e2}
    c_t + i_t &= r a_t + y_t \\
    a_{t+1}   &= a_t + i_t \\
    y _{t+1} &= \rho_1 y_t + \rho_2 y_{t-1}
  \end{align}

  Here $c_t, i_t, a_t$, and $ y_t$ are the household's consumption, investment, asset holdings, and exogenous labor income at $t$; while $b > 0$, $\gamma > 0$, $ r > 0$, $\beta \in (0, 1)$, ad $\rho_1,  \rho_2$ are parameters, and $a_0, y_0, y_{-1}$ are initial conditions. Assume that $\rho_1, \rho_2$ are such that $\left(1 - \rho_1 z - \rho_2 z^2 \right) =0$ implies that $|z| > 1$.

  \begin{enumerate}[a.]
    \item Map this problem into an optimal linear regulator problem.
    \item For parameter values $\left[ \beta, (1+ r), b, \gamma, \rho_1, \rho_2\right] = \left[0.95, 0.95 ^{-1}, 30, 1, 1.2, -.3 \right]$, compute the household's optimal policy function using your Matlab (Python) program from exercise 5.1.
  \end{enumerate}


  \vspace{.2in}

  \problemAnswer{

    \begin{enumerate}[a.]
      \item To map this into a linear regulator problem we need to come up with matrices $A, B, Q, $ and $R$ that explain the objective function and pin down the law of motion for the state variables.  Let the control vector and the state vector be the following:

        \begin{align} \label{eq:p3e3}
          u_t &= \begin{pmatrix} i_t \end{pmatrix} =\begin{pmatrix} a_{t+1} - a_t \end{pmatrix} \\
          x _t &= \begin{pmatrix}  a_t \\ y_t \\ y_{t-1}  \\ 1\end{pmatrix}
        \end{align}

        Now, I know that I am looking for the matrices $A$ and $B$ that satisfy $x_{t+1} = A x_t + B u_t$. Examining the given equations and using the definitions from equation \ref{eq:p3e3}, I can pin them down as follows:

        \begin{align} \label{eq:p3e4}
          A =
          \begin{bmatrix}
            1 & 0 & 0 & 0 \\
            0 & \rho_1 & \rho_2\ & 0  \\
            0 & 1 & 0 & 0 \\
            0 & 0 & 0 & 1
          \end{bmatrix}
          \quad
          B =
          \begin{bmatrix}
            1\\
            0\\
            0\\
            0
          \end{bmatrix}
        \end{align}

        In order to determine the values of $R$, $Q$, and $H$ I need to substitute out the $c_t$ in the objective function. Doing this yields the following:

        \begin{align} \label{eq:p3e5}
          & - \sum_{t=0}^{\infty} \beta^t \left\{ (c_t - b)^2 + \gamma i_t^2 \right \} \\
          & - \sum_{t=0}^{\infty} \beta^t \left\{ \left(r a_t-b-i_t+y_t\right){}^2+\gamma  i_t^2 \right \} \\
          & - \sum_{t=0}^{\infty} \beta^t \left\{ -2 b r a_t-2 r a_t i_t+r^2 a_t^2+2 r a_t y_t+b^2+2 b i_t-2 b y_t+\gamma  i_t^2-2 i_t y_t+i_t^2+y_t^2 \right\} \\
        \end{align}

        Using this form I am now ready to match patterns to define the matrices $R$, $Q$, and $H$ in the expression $x_t' R x_t + u_t'Qu_t + 2 u_t' H x_t $. To show my process, I will do this one item at a time, then show the entire matrices. For the $R$ matrix:

        \begin{itemize}
          \item $R_{1, 1}$: This is the coefficient on $a_t^2$. There is a $r^2a_t^2$, so this value is $r^2$.
          \item $R_{1, 2} = R_{2,1}$: This is 1/2 the coefficient on $y_t a_t$. I see that there is a $2r y_t a_t$ so this value is $r$.
          \item $R_{1, 4} = R_{4, 1}$: This is 1/2 the coefficient on $a_t$ by itself. There is a $-2br a_t$, so this value is $-br$.
          \item $R_{2, 2}$: This is the coefficient on $y_t^2$. I see that this must be 1.
          \item $R_{2, 4} = R_{4,2}$: This is 1/2 the coefficient on $y_t$ by itself. There is a $-2b y_t$ so this value is $-b$.
          \item $R_{:, 3} = R_{3, :} = 0$: The second row and column deal with $y_{t-1}$, which doesn't appear in the equation above.
          \item $R_{4, 4}$: This is the coefficient on $1$. There is a $b^2$, so this value is $b^2$.
        \end{itemize}

        The $Q$ matrix is the scalar coefficient on $i_t^2$, which is $(1 + \gamma)$.

        The $H$ matrix is explained below:

        \begin{itemize}
          \item $H_{1, 1}$: This is 1/2 the coefficient on $i_t a_t$. There is a  $-2 r i_t a_t$, so this value is $-r$.
          \item $H_{1, 2}$: This is 1/2 the coefficient on $i_t y_t$. There is a  $-2 i_t y_t$, so this value is $-1$.
          \item $H_{1, 3}$: This is 1/2 the coefficient on $i_t y_{t-1}$. There are not $y_{t-1}$, so this value is $0$.
          \item $H_{1, 4}$: This is 1/2 the coefficient on $i_t$ by itself. There is a  $2 b i_t$, so this value is $b$.
        \end{itemize}

        I am now ready to show all the matrices in their complete form.

        \begin{align} \label{eq:p3e6}
          R =
          \begin{bmatrix}
            r^2 & r & 0 & -br \\
            r & 1 & 0 & -b \\
            0 & 0 & 0 & 0 \\
            -br & -b & 0 & b^2 \\
          \end{bmatrix}
          \quad
          Q = (1 + \gamma)
          \quad
          H = \begin{bmatrix} -r & -1 & 0 & b \\ \end{bmatrix}
        \end{align}

        These matrices map the problem into an optimal linear regulator problem. Specifically the problem is to maximize with respect to $u$

        \begin{align} \label{eq:p3e7}
          - \sum_{t=0}^{\infty} \beta^t \{ x_t' R x_t + u_t'Qu_t + 2 u_t' H x_t \}
        \end{align}

        subject to the law of motion

        \begin{align} \label{eq:p3e8}
          x_{t+1} = Ax_t + B u_t
        \end{align}

      \item I used the function \texttt{riccati} from Problem 5.1 to compute the optimal policy $F$ in $u_{t+1} = -F x_t$ and the implied value of $P$. The code to produce the solution is contained in Listing \ref{code:p53b}. The solution I got was:

        $$F =
        \begin{bmatrix}
          -0.000 & -0.317 & -0.056 & 0.000\\
        \end{bmatrix}
        $$

        $$
        P =
        \begin{bmatrix}
          0.055 & 0.403 & -0.115 & -31.579\\
          0.403 & 4.591 & -1.250 & -229.446\\
          -0.115 & -1.250 & 0.386 & 65.392\\
          -31.579 & -229.446 & 65.392 & 18000.000\\
        \end{bmatrix}
        $$
    \end{enumerate}

    \setstretch{0.68}
    \lstinputlisting[language=Python, linerange={1-37}, caption= Solution to 5.3b, label=code:p53b]{../RMT4_ch5.py}
    \setstretch{1.5}
    \qed

  }
\end{homeworkProblem}
