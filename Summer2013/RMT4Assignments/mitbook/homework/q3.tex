%%%%%%%%%%%%%%

%%%%%%%%%%%%%%%%%%%%%%%%%%%%%%%%%%%%%%%%%%%%
%  q3.tex 
%  Teaching evaluation question 
%%%%%%%%%%%%%%%%%%%%%%%%%%%%%%%%%%%%%%%%%%%%
\input sargent
\magnification=1200   
\input grafinp3 
\input psfig 

\line{Stanford University \hfil Fall 1999}
\line{Economics 210 \hfil Thomas Sargent}




\centerline{\bf Teaching Evaluation: Part I}


\medskip
\noindent Without the required tools,
which were invented  only after World War II,
the following question is very difficult to answer.
Many of you have probably not yet learned those tools. 
Nevertheless, I ask you to spend some time thinking 
about the question.
  The question raises some of the substantive and 
analytic issues to be discussed in this course.
Please take an hour to think and write about
the following question.  Your answer will not be graded.
Please keep your answer until the end of the quarter, to compare 
it with the answer that you will give to another question
you will be asked then.

\vskip.1cm
\noindent
{\bf Part 1.} \ \ {\it The worker's problem.}
\medskip

\noindent Think about an economy in which workers  all confront the following common
environment. Time is discrete.  Let $t=0, 1, 2, \ldots$ index time. 
At the beginning of each period, a  previously employed
worker can choose to work at her last
period's wage or to draw a new wage. If she draws a new wage, the old wage is
lost and she can start working at the new wage in the following period.
New wages are independent and identically
distributed  (i.i.d.)  draws from the cumulative
distribution function $F$, where $F(0)=0$, $F(M)=1$
for $M<\infty$. Each unemployed worker
can draw one and only one wage per period from $F$.   Someone
offered a wage is  free
to work at that  wage for as long as she chooses (she cannot
be fired). 
The income of an unemployed worker is $b$, which includes
unemployment insurance and the value of home production. 
Each worker seeks to maximize 
$
E_0 \sum_{t=0}^{\infty} (1-\mu)^t \beta^t I_t,  
$
where $\mu$ is the probability that a worker dies at the end of a period,
$\beta$ is the subjective discount factor, and $I_t$ is the worker's income
in period $t$, i.e., $I_t$ is equal to the wage $w_t$ when employed and the 
income $b$ when unemployed.  Here $E_0$ is the mathematical
expectation operator, conditioned on information known at time $0$.
Assume that $\beta \in (0,1)$ and $\mu \in (0,1)$.

\medskip
\item{a.} Describe the  worker's optimal decision rule.  In particular,
what should an employed worker do?  What should an unemployed worker do?   
\vskip.1cm

\item{b.} How would an unemployed  worker's  
behavior be affected by an increase in $\mu$? 
\vskip.2cm 

\medskip
\noindent{\bf Part 2.}\ {\it  Equilibrium unemployment rate}
\medskip
\noindent
The economy is populated with a continuum of the workers just described.
 There  is an  exogenous
rate of new workers entering the labor market  equal to $\mu$, which
equals 
the death rate. New entrants are unemployed and must draw a new wage.

\vskip.2cm

\item{c.} 
 Discuss the
determinants of the unemployment rate.

\vskip.2cm 
\noindent
We now change the technology so that 
the economy fluctuates between booms ($B$) and recessions ($R$).
In a boom, all employed workers are paid an extra $z>0$.
That is, the income of a worker with wage $w$ is $I_t=w+z$ in a boom, and
$I_t=w$ in a recession.  Let whether the economy is
in a boom or a recession define the {\it state} of the economy.
Assume that  
the state of the economy is i.i.d.\ and that booms and
recessions have the same probabilities of
$0.5$. 

\vskip.2cm
\item{d. }  Describe the optimal behavior of employed and unemployed
 workers.  When, if ever, might workers choose to quit? 

\vskip.2cm
\item{e. } Discuss the determination of the unemployment rate. 

\vskip.1cm
\item{f. } The following time series is a simulation from
the solution of the model with booms   and recessions.  Interpret
the time series in terms of the model.

$$
\grafone{pic1.ps,height=2.1in}{\bf Unemployment rate }
$$




\end

